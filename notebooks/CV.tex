\documentclass[12pt,preprint,letter]{aastex63}
\usepackage{mycvstyle}
\pagestyle{CV}
\renewcommand{\baselinestretch}{0.97} % double-spaced
\setlength{\parskip}{0.02in}
\begin{document}
\begin{center}
{\Large \bf{ Curriculum Vitae --
%\vspace*{0.1cm}
Chang-Goo Kim} }
\end{center}
% \vspace{-10pt}
%%========================================================================================
Department of Astrophysical Sciences
\hfill +1-609-933-1180\\
Princeton University
\hfill \url{http://changgoo.github.io} \\
4 Ivy Lane, Princeton
\hfill \href{http://orcid.org/0000-0003-2896-3725}{ORCID: 0000-0003-2896-3725}\\
NJ 08544, USA
\hfill \url{cgkim@astro.princeton.edu}\\

%%========================================================================================
\itemtitle{Education}

\elements{Mar 2005-- \\Feb 2011}{Ph.~D in Astronomy}{Department of Physics and Astronomy, Seoul National University, Korea}
\elements{Mar 2001-- \\Feb 2005}{B.~S in Astronomy}{Department of Physics and Astronomy, Seoul National University, Korea}

%%========================================================================================
\itemtitle{Current Position}

\elements{Sep 2016 -- }{Associate Research Scholar}{Department of Astrophysical Sciences, Princeton University}

%%========================================================================================
\itemtitle{Employment}

\elements{Sep 2017 -- \\Aug 2018}{Flatiron Research Fellow}{Center for Computational Astrophysics, Flatiron Institute}
%\elements{Sep 2016 -- \\Aug 2017}{Associate Research Scholar}{Department of Astrophysical Sciences, Princeton University}
\elements{Sep 2013 -- \\Aug 2016}{Postdoctoral Research Associate}{Department of Astrophysical Sciences, Princeton University}
\elements{Oct 2011 -- \\Aug 2013}{CITA National Fellow}{Department of Physics and Astronomy, University of Western Ontario, Canada}
% \elements{Mar 2011 -- \\Aug 2011}{BK21 Postdoctoral Research Fellow}{Department of Physics and Astronomy, Seoul National University, Korea}

%%========================================================================================
\itemtitle{Grants}

\onelineelements{2022--2024}{PI}{NASA Astrophysics Theory Program; \$415,564}
\onelineelements{2019}{PI}{Chandra cycle 21 (Theory); \$85,000}
\onelineelements{2018--2021}{Co-I}{NASA TCAN (PI: Julian Borrill); \$1,398,099}

%%========================================================================================
\itemtitle{Research Advising}

\elements{2016 -- present}{PhD thesis projects}
{
  Woorak Choi (Yonsei, current),
  Sanghyuk Moon (SNU, PhD in 2022),
  Lachlan Lancaster (Princeton, PhD in 2022),
  Alwin Mao (Princeton, PhD in 2020),
  Munan Gong (Princeton, PhD in 2017)
}
\elements{2018 -- present}{Research projects for graduate students}
{
  Minghao Guo (Princeton, current),
  Nora Linzer (Princeton, current),
  Erin Kado-Fong (Princeton, 2018),
  Aditi Vijayan (CCA via \href{https://kspa.soe.ucsc.edu/archives/2018}{KSPA}, 2018),
  Kareem El-Badry (CCA via \href{https://kspa.soe.ucsc.edu/archives/2018}{KSPA}, 2018)
}
\elements{2014 -- present}{Research projects for undergraduate students}
{
  Ish Kaul (Princeton, current),
  Ryan Golant (Princeton, 2019),
  Mohammad Refat (CCA via \href{http://cunyastro.org/astrocom/}{AstroCom NYC}, 2018),
  Roberta Raileanu (Princeton, 2014)
}

\itemtitle{Teaching}

\elements{2021 -- 2022}{Bootcamp Lecturer, \textnormal{Undergraduate Summer Research Program in Astrophysics at Princeton University}}{- Teaching basic Unix commands and remote login (ssh), software version control (git and GitHub), Python programming language and scientific programming stack}
\elements{2005 -- 2010 }{Graduate Student Instructor (Teaching Assistant), \textnormal{Seoul National University}}{- Grading problem sets and leading problem-solving sessions for courses including \emph{Solar System Astronomy and Lab., Astronomical Observation \& Lab. I \& II, Astronomy and Lab., Introduction to Astrophysics I \& II, Stars and Stellar Systems, Man \& the Universe}. \\ - Designing and leading the Lab class for Introduction to Astronomy\\
%- Teaching programming languages and analysis tools including Fortran, C, and IDL. \\
- Teaching scientific computing and numerical analysis -- root-finding, numerical integration, linear algebra, linear regression }

%%========================================================================================
\itemtitle{Observing Proposals}

%\onelineelements{2021}{Co-I}{HST cycle 29 GO (PI: Erin Kado-Fong); submitted}
\onelineelements{2019}{Co-I}{VLA Extra Large proposal (PI: Adam Leroy); Local Group L-Band Survey}
\onelineelements{2019}{Co-I}{VLA Regular proposal (PI: Woorak Choi), 7.4 hours, rank B}

%%========================================================================================
\itemtitle{Computing Time Allocations}

\onelineelements{2022--2024}{15M CPU hrs (540k SBUs), PI}{NASA HECC}
\onelineelements{2018--2021}{80M CPU hrs, Co-I}{NERSC, (PI: Julian Borrill)}
\onelineelements{2016--2021}{24M CPU hrs (850k SBUs), Co-I}{NASA HECC, (PI: Eve Ostriker)}

%%========================================================================================
\itemtitle{Scientific Collaboration Teams}

\onelineelements{2022 -- }{Working Group Leader}{\href{https://www.learning-the-universe.org}{Simons Collaboration on Learning the Universe}}
%{\footnotesize Key member of  }
\onelineelements{2017 -- 2022}{Working Group Leader}{{\href{https://www.simonsfoundation.org/flatiron/center-for-computational-astrophysics/galaxy-formation/smaug/}{Simulating Multiscale Astrophysics to Understand Galaxies} (SMAUG)}}
%{\footnotesize leading the working group for ``Resolved ISM, Star formation, and Stellar feedback'' }
\onelineelements{2018 -- 2021}{Working Group Leader}{Modeling Polarized Galactic Foregrounds for Cosmic Microwave Background missions (NASA TCAN)}
%{\footnotesize leading the MHD simulation subnet in the multi-institutional collaboration funded by NASA entitled ``Modeling Polarized Galactic Foregrounds for CMB Missions'' }
\onelineelements{2022 -- }{Member}{\href{http://lem.cfa.harvard.edu}{Line Emission Mapper X-ray Probe}}
\onelineelements{2021 -- }{Member}{\href{https://www.lglbs.org}{Local Group L-Band Survey}}
%{\footnotesize 21-cm line and L-band continuum emission over the full area of the six actively star-forming Local Group galaxies using 1800 hours of ``L band'' (1-2 GHz) observations in all VLA configurations}
\onelineelements{2020 -- }{Member}{\href{https://gaskap.anu.edu.au}{Galactic Australian Square Kilometre Array Pathfinder Survey}}
%{\footnotesize high spectral resolution survey of the HI and OH lines in the Milky Way and Magellanic Systems }
\onelineelements{2019 -- 2020}{Member}{Space Infrared Telescope for Cosmology and Astrophysics (SPICA)}
%{\footnotesize member of the SPICA science case development team for ``Diffuse gas in galaxies'' }
\onelineelements{2017 -- 2019}{Member}{Probe of Inflation and Cosmic Origins (PICO)}
%{\footnotesize contributing galactic foreground modeling for a probe-class mission concept study funded by NASA entitled ``Probe of Inflation and Cosmic Origins''}

\itemtitle{Professional Service}

\onelineelements{2020 -- 2022}{Reviewer}{NASA FINESST}
\onelineelements{2017}{Review Panelist}{NSF AAG Program}
\onelineelements{2016 -- 2017}{Organizer}{Star Formation/ISM Rendezvous Seminars at Princeton University}
\onelineelements{2012 -- }{Referee}{ApJ, ApJL, MNRAS, JOSS}


% \itemtitle{}
%%========================================================================================
\itemtitle{References}

\begin{itemize}[itemsep=0pt,topsep=\parskip]
\item \subtitle{Eve C. Ostriker} \\
\url{eco@astro.princeton.edu}, +1-609-258-7240\\
Professor, Department of Astrophysical Sciences, Princeton University

\item \subtitle{Rachel S. Somerville} (co-sign with Prof. Bryan)\\
\url{rsomerville@flatironinstitute.org}, +1-848-445-8964\\
Group Leader, Center for Computational Astrophysics, Flatiron Institute

\item \subtitle{Greg L. Bryan} (co-sign with Prof. Somerville) \\
\url{gbryan@astro.columbia.edu}, +1-212-854-6837\\
%Group Leader, Center for Computational Astrophysics, Flatiron Institute\\
Professor, Department of Astronomy, Columbia University

\item \subtitle{James M. Stone} \\
\url{jmstone@ias.edu}, +1-609-734-8054\\
Professor, School of Natural Sciences, Institute for Advanced Study
% Emeritus Professor, Department of Astrophysical Sciences, Princeton University

%\item \subtitle{Woong-Tae Kim} \\
%\url{wkim@astro.snu.ac.kr}, +82-2-880-6769\\
%Professor, Department of Physics and Astronomy, Seoul National University

%\item \subtitle{Amiel Sternberg} \\
%\url{amiel@astro.tau.ac.il}, 03-6407590\\
%Professor, Department of Astronomy, Tel Aviv University
% Senior Research Scientist, Center for Computational Astrophysics, Flatiron Institute

%\item \subtitle{Raphael Flauger} \\
%\url{flauger@physics.ucsd.edu}, +1-858-534-7504\\
%Professor, Department of Physics, University of California, San Diego

\end{itemize}


\itemtitle{List of Publications}

% \begin{itemize}
% \begin{center}
% {\large \bf List of Publications} \\
% (\href{https://ui.adsabs.harvard.edu/search/q=\%3Dauthor\%3A\%22kim\%2Cchang-goo\%22}{ADS},
% \href{https://scholar.google.com/citations?user=jBOsJgoAAAAJ&hl=en}{Google Scholar})\\
{\student{Name}: student advised/co-advised by me}\\
    Metrics for Refereed Publications (from \href{\adsurl}{ADS} as of \textit{2022-11-04}) \\count: 43 --- citations: 1821 --- h-index: 23
% \end{center}

\itemtitle{I. Refereed Publications  as First Author (count: 17 --- citations: 1758)}
\begin{itemize}[itemsep=0pt,topsep=\parskip]
%\item \textbf{Five most significant contributions (\#20, 23, 26, 27, 29 marked by **):} I am the first and corresponding author of these papers. I led the projects, implemented numerical schemes, carried out the simulations on supercomputers in Canada (Sharcnet), Princeton (Tiger/Perseus), Flatiron Institute (Rusty), NASA (Pleiades), and NERSC (Cori),
%developed the analysis methods, created the plots and visualizations, and wrote the texts. This statement is true for all first-authored papers listed below except \#16, 17, and 18 for which Prof. Woong-Tae Kim was a corresponding author as the thesis supervisor. All co-authors contributed to the analysis and interpretation of the results, and the manuscript writing.
\item[{50.}]\textbf{Kim, Chang-Goo}; Kim, Jeong-Gyu; Gong, Munan; Ostriker, Eve C., \textit{Introducing TIGRESS-NCR. I. Coregulation of the Multiphase Interstellar Medium and Star Formation Rates}, \doiform{10.3847/1538-4357/acbd3a}{\apj}, \textbf{946}, 3, 2023 (\arxiv{2211.13293}) [\href{http://adsabs.harvard.edu/abs/2023ApJ...946....3K}{13 citations}]

\item[{49.}]\textbf{Kim, Chang-Goo}; Ostriker, Eve C.; Fielding, Drummond B.; Smith, Matthew C.~\textit{et al.}, \textit{A Framework for Multiphase Galactic Wind Launching Using TIGRESS}, \doiform{10.3847/2041-8213/abc252}{\apj}, \textbf{903}, 2020 (\arxiv{2010.09090}) [\href{http://adsabs.harvard.edu/abs/2020ApJ...903L..34K}{35 citations}]

\item[{48.}]\textbf{Kim, Chang-Goo}; Ostriker, Eve C.; Somerville, Rachel S.; Bryan, Greg L.~\textit{et al.}, \textit{First Results from SMAUG: Characterization of Multiphase Galactic Outflows from a Suite of Local Star-forming Galactic Disk Simulations}, \doiform{10.3847/1538-4357/aba962}{\apj}, \textbf{900}, 61, 2020 (\arxiv{2006.16315}) [\href{http://adsabs.harvard.edu/abs/2020ApJ...900...61K}{78 citations}]

\item[{47.}]\textbf{Kim, Chang-Goo}; Choi, Steve K.; Flauger, Raphael, \textit{Dust Polarization Maps from TIGRESS: E/B Power Asymmetry and TE Correlation}, \doiform{10.3847/1538-4357/ab29f2}{\apj}, \textbf{880}, 106, 2019 (\arxiv{1901.07079}) [\href{http://adsabs.harvard.edu/abs/2019ApJ...880..106K}{32 citations}]

\item[{46.}]\textbf{Kim, Chang-Goo}; Ostriker, Eve C., \textit{Numerical Simulations of Multiphase Winds and Fountains from Star-forming Galactic Disks. I. Solar Neighborhood TIGRESS Model}, \doiform{10.3847/1538-4357/aaa5ff}{\apj}, \textbf{853}, 173, 2018 (\arxiv{1801.03952}) [\href{http://adsabs.harvard.edu/abs/2018ApJ...853..173K}{154 citations}]

\item[{45.}]\textbf{Kim, Chang-Goo}; Ostriker, Eve C., \textit{Three-phase Interstellar Medium in Galaxies Resolving Evolution with Star Formation and Supernova Feedback (TIGRESS): Algorithms, Fiducial Model, and Convergence}, \doiform{10.3847/1538-4357/aa8599}{\apj}, \textbf{846}, 133, 2017 (\arxiv{1612.03918}) [\href{http://adsabs.harvard.edu/abs/2017ApJ...846..133K}{160 citations}]

\item[{44.}]\textbf{Kim, Chang-Goo}; Ostriker, Eve C.; Raileanu, Roberta, \textit{Superbubbles in the Multiphase ISM and the Loading of Galactic Winds}, \doiform{10.3847/1538-4357/834/1/25}{\apj}, \textbf{834}, 25, 2017 (\arxiv{1610.03092}) [\href{http://adsabs.harvard.edu/abs/2017ApJ...834...25K}{130 citations}]

\item[{43.}]\textbf{Kim, Chang-Goo}; Ostriker, Eve C., \textit{Vertical Equilibrium, Energetics, and Star Formation Rates in Magnetized Galactic Disks Regulated by Momentum Feedback from Supernovae}, \doiform{10.1088/0004-637X/815/1/67}{\apj}, \textbf{815}, 67, 2015 (\arxiv{1511.00010}) [\href{http://adsabs.harvard.edu/abs/2015ApJ...815...67K}{96 citations}]

\item[{42.}]\textbf{Kim, Chang-Goo}; Ostriker, Eve C., \textit{Momentum Injection by Supernovae in the Interstellar Medium}, \doiform{10.1088/0004-637X/802/2/99}{\apj}, \textbf{802}, 99, 2015 (\arxiv{1410.1537}) [\href{http://adsabs.harvard.edu/abs/2015ApJ...802...99K}{309 citations}]

\item[{41.}]\textbf{Kim, Chang-Goo}; Ostriker, Eve C.; Kim, Woong-Tae, \textit{Three-dimensional Hydrodynamic Simulations of Multiphase Galactic Disks with Star Formation Feedback. II. Synthetic H I 21 cm Line Observations}, \doiform{10.1088/0004-637X/786/1/64}{\apj}, \textbf{786}, 64, 2014 (\arxiv{1403.5566}) [\href{http://adsabs.harvard.edu/abs/2014ApJ...786...64K}{48 citations}]

\item[{40.}]\textbf{Kim, Chang-Goo}; Basu, Shantanu, \textit{Long-term Evolution of Decaying Magnetohydrodynamic Turbulence in the Multiphase Interstellar Medium}, \doiform{10.1088/0004-637X/778/2/88}{\apj}, \textbf{778}, 88, 2013 (\arxiv{1309.4996}) [\href{http://adsabs.harvard.edu/abs/2013ApJ...778...88K}{6 citations}]

\item[{39.}]\textbf{Kim, Chang-Goo}; Ostriker, Eve C.; Kim, Woong-Tae, \textit{Three-dimensional Hydrodynamic Simulations of Multiphase Galactic Disks with Star Formation Feedback. I. Regulation of Star Formation Rates}, \doiform{10.1088/0004-637X/776/1/1}{\apj}, \textbf{776}, 1, 2013 (\arxiv{1308.3231}) [\href{http://adsabs.harvard.edu/abs/2013ApJ...776....1K}{180 citations}]

\item[{38.}]\textbf{Kim, Chang-Goo}; Kim, Woong-Tae; Ostriker, Eve C., \textit{Regulation of Star Formation Rates in Multiphase Galactic Disks: Numerical Tests of the Thermal/Dynamical Equilibrium Model}, \doiform{10.1088/0004-637X/743/1/25}{\apj}, \textbf{743}, 25, 2011 (\arxiv{1109.0028}) [\href{http://adsabs.harvard.edu/abs/2011ApJ...743...25K}{134 citations}]

\item[{37.}]\textbf{Kim, Chang-Goo}; Kim, Woong-Tae; Ostriker, Eve C., \textit{Galactic Spiral Shocks with Thermal Instability in Vertically Stratified Galactic Disks}, \doiform{10.1088/0004-637X/720/2/1454}{\apj}, \textbf{720}, 1454, 2010 (\arxiv{1006.4691}) [\href{http://adsabs.harvard.edu/abs/2010ApJ...720.1454K}{23 citations}]

\item[{36.}]\textbf{Kim, Chang-Goo}; Kim, Woong-Tae; Ostriker, Eve C., \textit{Galactic Spiral Shocks with Thermal Instability}, \doiform{10.1086/588752}{\apj}, \textbf{681}, 1148, 2008 (\arxiv{0804.0139}) [\href{http://adsabs.harvard.edu/abs/2008ApJ...681.1148K}{55 citations}]

\item[{35.}]\textbf{Kim, Chang-Goo}; Kim, Woong-Tae; Ostriker, Eve C., \textit{Interstellar Turbulence Driving by Galactic Spiral Shocks}, \doiform{10.1086/508160}{\apj}, \textbf{649}, 2006 (\arxiv{astro-ph/0608161}) [\href{http://adsabs.harvard.edu/abs/2006ApJ...649L..13K}{45 citations}]
\end{itemize}

\smallskip
\itemtitle{II. Refereed Publications  w/ Significant Contribution (count: 18 --- citations: 318)}
\begin{itemize}[itemsep=0pt,topsep=\parskip]
\item[{30.}]\student{Moon, Sanghyuk}; Kim, Woong-Tae; \textbf{Kim, Chang-Goo}; Ostriker, Eve C., \textit{Effects of Magnetic Fields on Gas Dynamics and Star Formation in Nuclear Rings}, \doiform{10.3847/1538-4357/acc250}{\apj}, \textbf{946}, 114, 2023 (\arxiv{2303.04206})

\item[{29.}]\student{Guo, Minghao}; Stone, James M.; \textbf{Kim, Chang-Goo}; Quataert, Eliot, \textit{Toward Horizon-scale Accretion onto Supermassive Black Holes in Elliptical Galaxies}, \doiform{10.3847/1538-4357/acb81e}{\apj}, \textbf{946}, 26, 2023 (\arxiv{2211.05131}) [\href{http://adsabs.harvard.edu/abs/2023ApJ...946...26G}{6 citations}]

\item[{28.}]Kim, Jeong-Gyu; Gong, Munan; \textbf{Kim, Chang-Goo}; Ostriker, Eve C., \textit{Photochemistry and Heating/Cooling of the Multiphase Interstellar Medium with UV Radiative Transfer for Magnetohydrodynamic Simulations}, \doiform{10.3847/1538-4365/ac9b1d}{\apjs}, \textbf{264}, 10, 2023 (\arxiv{2210.08024}) [\href{http://adsabs.harvard.edu/abs/2023ApJS..264...10K}{7 citations}]

\item[{27.}]\student{Kado-Fong, Erin}; \textbf{Kim, Chang-Goo}; Greene, Jenny E.; Lancaster, Lachlan, \textit{Ultra-diffuse Galaxies as Extreme Star-forming Environments. II. Star Formation and Pressure Balance in H I-rich UDGs}, \doiform{10.3847/1538-4357/ac9673}{\apj}, \textbf{939}, 101, 2022 (\arxiv{2209.05500}) [\href{http://adsabs.harvard.edu/abs/2022ApJ...939..101K}{3 citations}]

\item[{26.}]Ostriker, Eve C.; \textbf{Kim, Chang-Goo}, \textit{Pressure-regulated, Feedback-modulated Star Formation in Disk Galaxies}, \doiform{10.3847/1538-4357/ac7de2}{\apj}, \textbf{936}, 137, 2022 (\arxiv{2206.00681}) [\href{http://adsabs.harvard.edu/abs/2022ApJ...936..137O}{25 citations}]

\item[{25.}]\student{Choi, Woorak}; \textbf{Kim, Chang-Goo}; Chung, Aeree, \textit{Ram Pressure Stripping of the Multiphase ISM: A Detailed View from TIGRESS Simulations}, \doiform{10.3847/1538-4357/ac82ba}{\apj}, \textbf{936}, 133, 2022 (\arxiv{2207.05263}) [\href{http://adsabs.harvard.edu/abs/2022ApJ...936..133C}{3 citations}]

\item[{24.}]\student{Moon, Sanghyuk}; Kim, Woong-Tae; \textbf{Kim, Chang-Goo}; Ostriker, Eve C., \textit{Effects of Varying Mass Inflows on Star Formation in Nuclear Rings of Barred Galaxies}, \doiform{10.3847/1538-4357/ac3a7b}{\apj}, \textbf{925}, 99, 2022 (\arxiv{2110.14882}) [\href{http://adsabs.harvard.edu/abs/2022ApJ...925...99M}{11 citations}]

\item[{23.}]\student{Lancaster, Lachlan}; Ostriker, Eve C.; Kim, Jeong-Gyu; \textbf{Kim, Chang-Goo}, \textit{Star Formation Regulation and Self-pollution by Stellar Wind Feedback}, \doiform{10.3847/2041-8213/ac3333}{\apj}, \textbf{922}, 2021 (\arxiv{2110.05508}) [\href{http://adsabs.harvard.edu/abs/2021ApJ...922L...3L}{18 citations}]

\item[{22.}]Clark, S. E.; \textbf{Kim, Chang-Goo}; Hill, J. Colin; Hensley, Brandon S., \textit{The Origin of Parity Violation in Polarized Dust Emission and Implications for Cosmic Birefringence}, \doiform{10.3847/1538-4357/ac0e35}{\apj}, \textbf{919}, 53, 2021 (\arxiv{2105.00120}) [\href{http://adsabs.harvard.edu/abs/2021ApJ...919...53C}{39 citations}]

\item[{21.}]\student{Lancaster, Lachlan}; Ostriker, Eve C.; Kim, Jeong-Gyu; \textbf{Kim, Chang-Goo}, \textit{Efficiently Cooled Stellar Wind Bubbles in Turbulent Clouds. II. Validation of Theory with Hydrodynamic Simulations}, \doiform{10.3847/1538-4357/abf8ac}{\apj}, \textbf{914}, 90, 2021 (\arxiv{2104.07722}) [\href{http://adsabs.harvard.edu/abs/2021ApJ...914...90L}{43 citations}]

\item[{20.}]\student{Lancaster, Lachlan}; Ostriker, Eve C.; Kim, Jeong-Gyu; \textbf{Kim, Chang-Goo}, \textit{Efficiently Cooled Stellar Wind Bubbles in Turbulent Clouds. I. Fractal Theory and Application to Star-forming Clouds}, \doiform{10.3847/1538-4357/abf8ab}{\apj}, \textbf{914}, 89, 2021 (\arxiv{2104.07691}) [\href{http://adsabs.harvard.edu/abs/2021ApJ...914...89L}{60 citations}]

\item[{19.}]\student{Moon, Sanghyuk}; Kim, Woong-Tae; \textbf{Kim, Chang-Goo}; Ostriker, Eve C., \textit{Star Formation in Nuclear Rings with the TIGRESS Framework}, \doiform{10.3847/1538-4357/abfa93}{\apj}, \textbf{914}, 9, 2021 (\arxiv{2104.10349}) [\href{http://adsabs.harvard.edu/abs/2021ApJ...914....9M}{14 citations}]

\item[{18.}]Koo, Bon-Chul; \textbf{Kim, Chang-Goo}; Park, Sangwook; Ostriker, Eve C., \textit{Radiative Supernova Remnants and Supernova Feedback}, \doiform{10.3847/1538-4357/abc1e7}{\apj}, \textbf{905}, 35, 2020 (\arxiv{2011.06322}) [\href{http://adsabs.harvard.edu/abs/2020ApJ...905...35K}{13 citations}]

\item[{17.}]Gong, Munan; Ostriker, Eve C.; \textbf{Kim, Chang-Goo}; Kim, Jeong-Gyu, \textit{The Environmental Dependence of the XCO Conversion Factor}, \doiform{10.3847/1538-4357/abbdab}{\apj}, \textbf{903}, 142, 2020 (\arxiv{2009.14631}) [\href{http://adsabs.harvard.edu/abs/2020ApJ...903..142G}{46 citations}]

\item[{16.}]Seon, Kwang-il; \textbf{Kim, Chang-Goo}, \textit{Ly-alpha Radiative Transfer: Monte Carlo Simulation of the Wouthuysen-Field Effect}, \doiform{10.3847/1538-4365/aba2d6}{\apjs}, \textbf{250}, 9, 2020 (\arxiv{2005.00238}) [\href{http://adsabs.harvard.edu/abs/2020ApJS..250....9S}{22 citations}]

\item[{15.}]\student{Mao, S. Alwin}; Ostriker, Eve C.; \textbf{Kim, Chang-Goo}, \textit{Cloud Properties and Correlations with Star Formation in Self-consistent Simulations of the Multiphase ISM}, \doiform{10.3847/1538-4357/ab989c}{\apj}, \textbf{898}, 52, 2020 (\arxiv{1911.05078}) [\href{http://adsabs.harvard.edu/abs/2020ApJ...898...52M}{20 citations}]

\item[{14.}]Kim, Woong-Tae; \textbf{Kim, Chang-Goo}; Ostriker, Eve C., \textit{Local Simulations of Spiral Galaxies with the TIGRESS Framework. I. Star Formation and Arm Spurs/Feathers}, \doiform{10.3847/1538-4357/ab9b87}{\apj}, \textbf{898}, 35, 2020 (\arxiv{2006.05614}) [\href{http://adsabs.harvard.edu/abs/2020ApJ...898...35K}{38 citations}]

\item[{13.}]\student{Kado-Fong, Erin}; Kim, Jeong-Gyu; Ostriker, Eve C.; \textbf{Kim, Chang-Goo}, \textit{Diffuse Ionized Gas in Simulations of Multiphase, Star-forming Galactic Disks}, \doiform{10.3847/1538-4357/ab9abd}{\apj}, \textbf{897}, 143, 2020 (\arxiv{2006.06697}) [\href{http://adsabs.harvard.edu/abs/2020ApJ...897..143K}{22 citations}]

\item[{12.}]\student{Vijayan, Aditi}; \textbf{Kim, Chang-Goo}; Armillotta, Lucia; Ostriker, Eve C.~\textit{et al.}, \textit{Kinematics and Dynamics of Multiphase Outflows in Simulations of the Star-forming Galactic Interstellar Medium}, \doiform{10.3847/1538-4357/ab8474}{\apj}, \textbf{894}, 12, 2020 (\arxiv{1911.07872}) [\href{http://adsabs.harvard.edu/abs/2020ApJ...894...12V}{24 citations}]

\item[{11.}]\student{El-Badry, Kareem}; Ostriker, Eve C.; \textbf{Kim, Chang-Goo}; Quataert, Eliot~\textit{et al.}, \textit{Evolution of supernovae-driven superbubbles with conduction and cooling}, \doiform{10.1093/mnras/stz2773}{\mnras}, \textbf{490}, 1961, 2019 (\arxiv{1902.09547}) [\href{http://adsabs.harvard.edu/abs/2019MNRAS.490.1961E}{49 citations}]

\item[{10.}]Gong, Munan; Ostriker, Eve C.; \textbf{Kim, Chang-Goo}, \textit{The X CO Conversion Factor from Galactic Multiphase ISM Simulations}, \doiform{10.3847/1538-4357/aab9af}{\apj}, \textbf{858}, 16, 2018 (\arxiv{1803.09822}) [\href{http://adsabs.harvard.edu/abs/2018ApJ...858...16G}{51 citations}]
\end{itemize}

\smallskip
\itemtitle{III. Refereed Publications   as Co-Author (count: 9 --- citations: 327)}
\begin{itemize}[itemsep=0pt,topsep=\parskip]
\item[{9.}]Motwani, Bhawna; Genel, Shy; Bryan, Greg L.; \textbf{Kim, Chang-Goo}~\textit{et al.}, \textit{First Results from SMAUG: Insights into Star Formation Conditions from Spatially Resolved ISM Properties in TNG50}, \doiform{10.3847/1538-4357/ac3d2d}{\apj}, \textbf{926}, 139, 2022 (\arxiv{2006.16314}) [\href{http://adsabs.harvard.edu/abs/2022ApJ...926..139M}{9 citations}]

\item[{8.}]Pingel, N. M.~\textit{et al.}~(incl. \textbf{CGK}), \textit{GASKAP-HI pilot survey science I: ASKAP zoom observations of HI emission in the Small Magellanic Cloud}, \doiform{10.1017/pasa.2021.59}{\pasa}, \textbf{39}, 2022 (\arxiv{2111.05339}) [\href{http://adsabs.harvard.edu/abs/2022PASA...39....5P}{8 citations}]

\item[{7.}]Pandya, V.~\textit{et al.}~(incl. \textbf{CGK}), \textit{Characterizing mass, momentum, energy, and metal outflow rates of multiphase galactic winds in the FIRE-2 cosmological simulations}, \doiform{10.1093/mnras/stab2714}{\mnras}, \textbf{508}, 2979, 2021 (\arxiv{2103.06891}) [\href{http://adsabs.harvard.edu/abs/2021MNRAS.508.2979P}{53 citations}]

\item[{6.}]Pandya, V.~\textit{et al.}~(incl. \textbf{CGK}), \textit{First Results from SMAUG: The Need for Preventative Stellar Feedback and Improved Baryon Cycling in Semianalytic Models of Galaxy Formation}, \doiform{10.3847/1538-4357/abc3c1}{\apj}, \textbf{905}, 4, 2020 (\arxiv{2006.16317}) [\href{http://adsabs.harvard.edu/abs/2020ApJ...905....4P}{29 citations}]

\item[{5.}]Fielding, D. B.~\textit{et al.}~(incl. \textbf{CGK}), \textit{First Results from SMAUG: Uncovering the Origin of the Multiphase Circumgalactic Medium with a Comparative Analysis of Idealized and Cosmological Simulations}, \doiform{10.3847/1538-4357/abbc6d}{\apj}, \textbf{903}, 32, 2020 (\arxiv{2006.16316}) [\href{http://adsabs.harvard.edu/abs/2020ApJ...903...32F}{40 citations}]

\item[{4.}]Murray, Claire E.; Peek, J. E. G.; \textbf{Kim, Chang-Goo}, \textit{Extracting the Cold Neutral Medium from H I Emission with Deep Learning: Implications for Galactic Foregrounds at High Latitude}, \doiform{10.3847/1538-4357/aba19b}{\apj}, \textbf{899}, 15, 2020 (\arxiv{2006.16490}) [\href{http://adsabs.harvard.edu/abs/2020ApJ...899...15M}{21 citations}]

\item[{3.}]Murray, C. E.~\textit{et al.}~(incl. \textbf{CGK}), \textit{The 21-SPONGE H I Absorption Line Survey. I. The Temperature of Galactic H I}, \doiform{10.3847/1538-4365/aad81a}{\apjs}, \textbf{238}, 14, 2018 (\arxiv{1806.06065}) [\href{http://adsabs.harvard.edu/abs/2018ApJS..238...14M}{73 citations}]

\item[{2.}]Murray, Claire E.; Stanimirovi{\'c}, Sne{\v{z}}ana; \textbf{Kim, Chang-Goo}; Ostriker, Eve C.~\textit{et al.}, \textit{Recovering Interstellar Gas Properties with Hi Spectral Lines: A Comparison between Synthetic Spectra and 21-SPONGE}, \doiform{10.3847/1538-4357/aa5d12}{\apj}, \textbf{837}, 55, 2017 (\arxiv{1612.02017}) [\href{http://adsabs.harvard.edu/abs/2017ApJ...837...55M}{22 citations}]

\item[{1.}]Safranek-Shrader, Chalence; Krumholz, Mark R.; \textbf{Kim, Chang-Goo}; Ostriker, Eve C.~\textit{et al.}, \textit{Chemistry and radiative shielding in star-forming galactic discs}, \doiform{10.1093/mnras/stw2647}{\mnras}, \textbf{465}, 885, 2017 (\arxiv{1605.07618}) [\href{http://adsabs.harvard.edu/abs/2017MNRAS.465..885S}{47 citations}]
\end{itemize}

% \itemtitle{Papers under Review}
% \begin{itemize}[itemsep=1pt,topsep=\parskip]
% \item \student{Guo, Minghao}; Stone, James M.; \textbf{Kim, Chang-Goo}; Quataert, Eliot, \textit{Toward Horizon-scale Accretion Onto Supermassive Black Holes in Elliptical Galaxies}, 2022 (\arxiv{2211.05131}), ApJ submitted
% \end{itemize}

\smallskip
\itemtitle{IV. Papers in preparation}
\begin{itemize}[itemsep=0pt,topsep=\parskip]
\item \boldname{}, Jeong-Gyu Kim, Munan Gong, Eve C. Ostriker, \textit{Introducing the TIGRESS-NCR Framework: II. The Relative Importance of Supernovae, Radiation, and Magnetic Fields in Regulating Star Formation Rates}
\item \boldname{}, Jeong-Gyu Kim, Munan Gong, Eve C. Ostriker, \textit{Introducing the TIGRESS-NCR Framework: III. Energetics of the ISM}
\item \student{Lachlan Lancaster}, Eve C. Ostriker, Jeong-Gyu Kim, \boldname{}, \textit{The Effects of Magnetic Fields and Photoionized Gas on the Dynamics of Stellar Wind Bubbles}
\item \student{Minghao Guo}, \boldname{}, James M. Stone, Eliot Quataert, \textit{The Effects of Dynamical and Thermal Evaporation of Cold Clouds in Supernova Remnants}
\item \student{Nora Linzer}, Eve C. Ostriker, Jeong-Gyu Kim, \boldname{}, \textit{Radiation Fields in TIGRESS-NCR}
\item Lucia Armillotta, Eve C. Ostriker, \boldname{}, Yan-Fei Jiang, \textit{Numerical Simulations of
the Effects of Cosmic Rays on Acceleration of Hot and Warm Galactic Outflows}
\item Ulrich P. Steinwandel \& \boldname{} et al., \textit{The Structure and Composition of Multiphase Galactic Winds in a Large Magellanic Cloud Mass Simulated Galaxy}
\item Sultan Hassan \& \boldname{} et al., \textit{Comparing star formation in TNG50 with the  Pressure-Regulated, Feedback-Modulated Equilibrium Model}
\item Matthew C. Smith \& \boldname{} et al., \textit{Arkenstone I: A Novel Method for Robustly Capturing High Specific Energy Outflows In (Pseudo)-Lagrangian Cosmological Simulations}
\end{itemize}

\itemtitle{V. Conference Proceedings}
\begin{itemize}[itemsep=0pt,topsep=\parskip]
\item \textbf{Kim, Chang-Goo}; Ostriker, Eve C., 2016 (\arxiv{1511.00018}), In P.~{Jablonka},
  P.~{Andr{\'e}}, and F.~{van der Tak}, editors, {\em From Interstellar Clouds
  to Star-Forming Galaxies: Universal Processes?}, volume 315 of {\em IAU
  Symposium}, pages 38--41, \doiform{10.1017/S1743921316007225}{Feedback Regulated Turbulence, Magnetic
  Fields, and Star Formation Rates in Galactic Disks}.
\item \textbf{Kim, Chang-Goo}; Ostriker, Eve C.; Kim, Woong-Tae, 2015 (\arxiv{1211.5161}),
  Highlights of Astronomy, 16:609--610, March 2015, \doiform{10.1017/S174392131401240X}{Numerical modeling
  of multiphase, turbulent galactic disks with star formation feedback}.
\end{itemize}

%%========================================================================================
\smallskip
% \newpage
% \begin{center}
% {\large \bf List of Professional Presentations}
% \end{center}

\itemtitle{List of Professional Presentations}

\itemtitle{I. Invited Review Talks}

\onelineelements{11/2019}{Invited Review}{\emph{Feedback Regulated Star Formation}, \href{https://mist2019.sciencesconf.org/resource/page/id/2}{Cosmic turbulence and magnetic fields: physics of baryonic matter across time and scales}, Carg\'ese, France}
\onelineelements{3/2019}{Invited Review}{
\emph{Galactic Star Formation Rates},
\href{https://www.aao.gov.au/conference/australia-eso-conference-2019}{Linking galaxies from the Epoch of initial star-formation to today}, Sydney, Australia}
\onelineelements{8/2016}{Invited Review}{
\emph{How Do Supernovae Regulate Star Formation and Launch Galactic Winds?},
\href{https://agenda.albanova.se/conferenceDisplay.py?confId=5696}{How Galaxies Form Stars}, Stockholm, Sweden}

\itemtitle{II. Invited Colloquia}

\onelineelements{11/2022}{Colloquium}{\emph{TBD}, University of Wisconsin-Madison, Madison, WI}
\onelineelements{11/2022}{Colloquium}{\emph{TBD}, Osaka University, Osaka, Japan}
\onelineelements{8/2022}{Colloquium}{\emph{Numerical modeling of the star-forming ISM:  SFRs, Outflows, and ISM energetics}, Korea Astronomy and Space Science Institute, Daejeon, Korea}
\onelineelements{4/2022}{Colloquium}{\emph{Galactic Star Formation Rates and Multiphase Outflow Driving in the Star-Forming ISM}, University of Florida, Gainesville, FL}
\onelineelements{3/2020}{Colloquium}{\emph{Self-Regulation of Star Formation Rates and Launching of Multiphase Galactic Winds}, University of Georgia, Athens, GA}
\onelineelements{2/2020}{Colloquium}{\emph{Self-Regulation of Star Formation Rates and Launching of Multiphase Galactic Winds}, University of Waterloo, Waterloo, ON, Canada}
\onelineelements{3/2019}{Colloquium}{\emph{Introducing TIGRESS: Where Gravity and Feedback Meet the Real ISM}, University of Maryland, College Park, MD}
\onelineelements{2/2019}{Colloquium}{\emph{Introducing TIGRESS: Where Gravity and Feedback Meet the Real ISM}, Australia National University, Canberra, Australia}
\onelineelements{8/2018}{Colloquium}{\emph{Star Formation Rates and Galactic Winds in TIGRESS}, Yonsei University, Seoul, Korea}
\onelineelements{8/2018}{Colloquium}{\emph{Star Formation Rates and Galactic Winds in TIGRESS}, Korea Astronomy and Space Science Institute, Daejeon, Korea}
\onelineelements{5/2017}{Colloquium}{\emph{Supernova as a Powerful Regulator of Galactic SFRs and Winds}, Osaka University, Osaka, Japan}
\onelineelements{2/2017}{Colloquium}{\emph{Galactic Star Formation Rates Regulated by Star Formation Feedback}, University of California, Santa Barbara, CA}
\onelineelements{10/2016}{Colloquium}{\emph{Self-Regulation of Star Formation Rates in Galactic Disks}, Shanghai Jiao Tong University, Shanghai, China}
\onelineelements{10/2016}{Colloquium}{\emph{Supernova Driven Galactic Outflows}, Korea Astronomy and Space Science Institute, Daejeon, Korea}
\onelineelements{10/2016}{Colloquium}{\emph{Supernova Driven Galactic Outflows}, Seoul National University, Seoul, Korea}
\onelineelements{9/2014}{Colloquium}{\emph{Supernova Feedback in Multiphase Galactic Disks}, Seoul National University, Seoul, Korea}
\onelineelements{9/2014}{Colloquium}{\emph{Supernova Feedback in Multiphase Galactic Disks}, Korea Astronomy and Space Science Institute, Daejeon, Korea}
\onelineelements{9/2014}{Colloquium}{\emph{Supernova Feedback in Multiphase Galactic Disks}, Korea Institute for Advanced Study, Seoul, Korea}
\onelineelements{9/2011}{Colloquium}{\emph{Regulation of Star Formation Rates in Galactic Disks}, Yonsei University, Seoul, Korea}
\onelineelements{3/2011}{Colloquium}{\emph{Thermal and Dynamical Evolution of a Gaseous Medium and Star Formation in Disk Galaxies}, National Institute for Mathematical Sciences, Daejeon, Korea}

\itemtitle{III. Conference/Workshop/Seminar Talks}

\onelineelements{8/2022}{Contributed Talk}{\emph{How Are Galactic Star Formation Rates Regulated?}, IAU Symposium \#373: Resolving the Rise and Fall of Star Formation in Galaxies, Busan, Korea}
\onelineelements{7/2022}{Invited Talk}{\emph{Introducing TIGRESS-NCR:
ISM energetics/phases and SFRs}, Interstellar Institute \#5: With Two Eyes, Orsay, France}
\onelineelements{7/2022}{Contributed Talk}{\emph{How Are Galactic Star Formation Rates Regulated?}, A Holistic View of Stellar Feedback and Galaxy Evolution, Ascona, Switzerland}
\onelineelements{5/2022}{Invited Talk}{\emph{How Do Stellar Feedback Regulates Galactic Star Formation Rates and Drives Multiphase Outflows?}, Theory Seminar, CITA, Toronto, Canada}
\onelineelements{10/2021}{Invited Talk}{\emph{How Are Galactic Star Formation Rates Regulated?}, Star Formation Lunch Seminar, CEA-Saclay, Paris, France}
\onelineelements{8/2021}{Invited Talk}{\emph{Multiphase Galactic Outflows in
TIGRESS}, Baltimore Wind Workshop 2021, Baltimore, MD}
\onelineelements{6/2021}{Invited Talk}{\emph{The Role of Magnetic Fields in Regulating Star Formation Rates}, Midwest Magnetic Field Meeting 2021, Madison, WI (remote)}
\onelineelements{4/2021}{Invited Talk}{\emph{MHD Simulations of the ISM and Synthetic Dust Polarization Maps}, Pan-Experiment Galactic Science Group Seminar, \href{https://galsci.github.io}, remote}
% \onelineelements{11/2020}{Invited Talk}{\emph{Multiphase Outflows in TIGRESS}, Peng Oh's Group Meeting, University of Santa Barbara, remote}
\onelineelements{8/2020}{Invited Talk}{\emph{A Perspective on the Future of ISM Simulations in the 2030s}, Cosmology with CMB-S4, University of Chicago, remote}
% \honelineelements{2020}{Invited Talk}{\href{https://caffelattes.sciencesconf.org}{Cosmological Analyses Featuring Galactic Foreground Emission}, Lattes, France -- canceled due to the pandemic}
\onelineelements{6/2019}{Contributed Talk}{\emph{Multiphase Outflows in TIGRESS}, Feedback and its Role in Galaxy Formation, Spetses, Greece}
% \onelineelements{6/2019}{Poster}{\emph{Dust Polarization Maps from TIGRESS}, Linking the Milky Way and Nearby Galaxies, Helsinki, Finland}
\onelineelements{3/2019}{Invited Talk}{\emph{Fast Fourier Transform and Self Gravity}, Athena++ Workshop 2019, UNLV, Las Vegas, NV}
\onelineelements{2/2019}{Invited Talk}{\emph{Multiphase ISM interacting with ICM}, Multi-phase Gas Workshop, CCA, New York, NY}
\onelineelements{10/2018}{Invited Talk}{\emph{Synthetic Observations of TIGRESS: Dust Polarization Maps, HI 21cm Lines, and more}, The Milky Way in the age of Gaia, Orsay, France}
\onelineelements{9/2018}{Contributed Talk}{\emph{Galactic Winds in TIGRESS}, THINKSHOP15, Potsdam, Germany}
\onelineelements{7/2018}{Invited Talk}{\emph{Star Formation Rates and Galactic Winds in TIGRESS}, Kavli Summer Program in Astrophysics, CCA, New York, NY}
\onelineelements{6/2018}{Invited Talk}{\emph{Synthetic Polarized Dust Emission from Self-Consistent MHD Simulations}, CMB Foreground Workshop at CCA, New York, NY}
\onelineelements{4/2018}{Invited Talk}{\emph{Partner of Cosmic Rays: Multiphase ISM and Galactic Outflows}, MPPC Workshop, Princeton, NJ}
\onelineelements{3/2018}{Invited Talk}{\emph{Star Formation and Galactic Winds in Self-Consistent Local ISM Simulations}, Computational Galaxy Formation at Ringberg Castle, Tegernsee, Germany}
\onelineelements{11/2017}{Invited Talk}{\emph{Self-Consistent MHD Simulations of the Local ISM:
Synthetic Polarized Dust Emission}, CMB Foreground Workshop at UCSD, San Diego, CA}
\onelineelements{7/2017}{Invited Talk}{\emph{TIGRESS: Three-phase ISM in Galaxies Resolving Evolution with Star formation and Supernova feedback}, The ISM beyond 3D, Orsay, France}
\onelineelements{2/2017}{Invited Talk}{\emph{Supernova Driven Galactic Winds and Synthetic Observations using TIGRESS}, Astrophysics Seminar, UCSB, Santa Barbara, CA}
\onelineelements{10/2016}{Invited Talk}{\emph{How do Supernovae Regulate Star Formation and Launch Galactic Winds?}, 7th East-Asia Numerical Astrophysics Meeting, Beijing, China}
\onelineelements{5/2016}{Invited Talk}{\emph{Star Formation and Galactic Winds Regulated by Supernovae}, Computational Galaxy Formation at Ringberg Castle, Tegernsee, Germany}
\onelineelements{10/2015}{Contributed Talk}{\emph{Generation and Saturation of Magnetic Fields in the ISM Regulated by Star Formation Feedback}, Magnetic Fields in the Universe V, Carg\'ese, France}
\onelineelements{8/2015}{Contributed Talk}{\emph{Feedback Regulated Turbulence, Magnetic Fields, and SFRs in Galactic Disks}, IAU Symposium \#315, Honolulu, HI}
\onelineelements{4/2015}{Invited Talk}{\emph{Feedback Regulated Turbulence, Magnetic Fields, and SFRs in Galactic Disks}, IAS Informal Seminar, IAS, Princeton, NJ}
\onelineelements{9/2014}{Invited Talk}{\emph{Feedback Regulated SFRs and HI 21cm Lines}, 6th East-Asia Numerical Astrophysics Meeting, Suwon, Korea}
\onelineelements{6/2014}{Invited Talk}{\emph{Momentum Injection by Supernovae in the ISM}, KITP Program -- Gravity's Loyal Opposition, Santa Barbara, CA}
\onelineelements{4/2013}{Contributed Talk}{\emph{Long-Term Evolution of Decaying MHD Turbulence in the Multiphase ISM}, KAS Spring Meeting, Daecheon, Korea}
\onelineelements{2/2013}{Invited Talk}{\emph{Long-Term Evolution of Decaying MHD Turbulence in the Multiphase ISM}, CITA National Fellow Meeting, Toronto, Canada}
\onelineelements{1/2013}{Contributed Talk}{\emph{Long-Term Evolution of Decaying MHD Turbulence in the Multiphase ISM}, AAS Meeting \#221, Long Beach, CA}
\onelineelements{8/2012}{Invited Talk}{\emph{Numerical Modeling of Multiphase, Turbulent Galactic Disks with Star Formation Feedback}, IAU General Assembly -- SpS12, Beijing, China}

% \item \subtitle{Snezana Stanimirovi\'c} --
% \url{sstanimi@astro.wisc.edu}, +1-608-890-1458\\
% Professor, Department of Astronomy, University of Wisconsin-Madison



\end{document}
