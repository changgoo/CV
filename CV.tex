\documentclass[12pt,preprint,letterpaper]{aastex62}
\usepackage{mycvstyle}
\pagestyle{CV}
\begin{document}
\begin{center}
{\huge \bf Curriculum Vitae}\\
%\vspace*{0.1cm}
{Chang-Goo Kim (cgkim@astro.princeton.edu)}
\end{center}

%%========================================================================================
Department of Astrophysical Sciences 
\hfill +1-609-933-1180\\
Princeton University 
\hfill \url{http://changgoo.github.io} \\
4 Ivy Lane, Princeton 
\hfill \href{http://orcid.org/0000-0003-2896-3725}{ORCID: 0000-0003-2896-3725}\\
NJ 08544, USA
\hfill \url{cgkim@astro.princeton.edu}\\

%%========================================================================================
\itemtitle{Current Position}

\elements{Sep 2018 -- }{Associate Research Scholar}{Department of Astrophysical Sciences, Princeton University}

%%========================================================================================
\itemtitle{Employment}

\elements{Sep 2017 -- \\Aug 2018}{Flatiron Research Fellow}{Center for Computational Astrophysics (CCA), Flatiron Institute}

\elements{Sep 2016 -- \\Aug 2017}{Associate Research Scholar}{Department of Astrophysical Sciences, Princeton University}

\elements{Sep 2013 -- \\Aug 2016}{Postdoctoral Research Associate}{Department of Astrophysical Sciences, Princeton University}

\elements{Oct 2011 -- \\Aug 2013}{CITA National Fellow}{Department of Physics and Astronomy, University of Western Ontario, Canada}

\elements{Mar 2011 -- \\Aug 2011}{BK21 Postdoctoral Research Fellow}{Department of Physics and Astronomy, Seoul National University, Korea}

%%========================================================================================
\itemtitle{Education}

\elements{Mar 2005-- \\Feb 2011}{Ph.~D in Astronomy}{Department of Physics and Astronomy, Seoul National University, Korea}

\elements{Mar 2001-- \\Feb 2005}{B.~S in Astronomy}{Department of Physics and Astronomy, Seoul National University, Korea}

%%========================================================================================
\itemtitle{Teaching Experience}

\elements{2019 -- \emph{present}}{Ryan Golant, \textnormal{Undergraduate student at Princeton University}}{Summer research (with Eve Ostriker)}

\elements{2018 -- \emph{present}}{Erin Kado-Fong, \textnormal{Graduate student at Princeton University}}{Semester project (with Eve Ostriker, Jeong-Gyu Kim)}

\elements{2018}{Aditi Vijayan, \textnormal{Graduate student at the Indian Institute of Science}}{Summer research via \href{https://kspa.soe.ucsc.edu/2018}{Kavli Summer Program in Astrophysics} (with Eve Ostriker, Lucia Armillotta, Miao Li)}

\elements{2018}{Kareem El-Badry, \textnormal{Graduate student at the UC Berkeley}}{Summer research via \href{https://kspa.soe.ucsc.edu/2018}{Kavli Summer Program in Astrophysics} (with Eve Ostriker)}

\elements{2018}{Mohammad Refat, \textnormal{Undergraduate student at the CUNY}}{Summer research via \href{http://cunyastro.org/astrocom/}{AstroCom NYC}}

\elements{2018 -- \emph{present}}{Erin Flowers, \textnormal{Graduate student at Princeton University}}{Semester project (with Eve Ostriker)}

\elements{2017 -- \emph{present}}{Woorak Choi, \textnormal{Graduate student at Yonsei University}}{Ph.D thesis project (with Aeree Chung)}

\elements{2014 -- 2015 }{Roberta Raileanu, \textnormal{Undergraduate student at Princeton University}}{Junior Thesis and Summer research (with Eve Ostriker)}

\elements{2005 -- 2010 }{Teaching Assistant, \textnormal{Seoul National University}}{\footnotesize Grading problem sets and leading problem-solving sessions for courses including \emph{Solar System Astronomy and Lab., Astronomical Observation \& Lab. I \& II, Astronomy and Lab., Introduction to Astrophysics I \& II, Stars and Stellar Systems, Man \& the Universe}. \\ Designing and leading the Lab classes. \\
Teaching programming languages and analysis tools including Fortran, C, and IDL.}

%%========================================================================================
\itemtitle{Grants}

\onelineelements{2020--2022}{PI}{NASA ATP (submitted); \$409,071}
\onelineelements{2020}{PI}{Hubble Theory Grant (submitted); \$150,000}
\onelineelements{2020}{PI}{Chandra Theory Grant (submitted); \$85,000}
\onelineelements{2018--2021}{Co-I}{NASA TCAN (PI: Julian Borrill); \$1,398,099}

%%========================================================================================
\itemtitle{Computing Time Allocations}

\onelineelements{2020--2022}{33M CPU hrs (1.2M SBUs), PI}{NASA Pleiades (submitted)}
\onelineelements{2019}{60M CPU hrs, Co-I}{ASCR Leadership Computing Challenge (submitted; PI: Alex Lazarian)}
\onelineelements{2018--2021}{80M CPU hrs, Co-I}{NERSC, (PI: Julian Borrill)}
\onelineelements{2016--2019}{22M CPU hrs (800k SBUs), Co-I}{NASA Pleiades, (PI: Eve Ostriker)}

%%========================================================================================
\itemtitle{Professional Activities and Services}

\elements{2018 -- 2021}{Subnet Leader, \textnormal{NASA TCAN}}{\footnotesize 
leading the MHD simulation subnet in the multi-institutional collaboration funded by NASA entitled ``Modeling Polarized Galactic Foregrounds for CMB Missions''
%consists of four node institutions (Princeton, UC San Diego, UC Berkeley, and U. of Wisconsin-Madison) 
%with three subnets (MHD simulation, foreground modeling, and data synthesis).\\
%I'm leading the subnet for MHD simulations.
}

\elements{2017 -- 2022}{Working Group Leader, \textnormal{\href{https://www.simonsfoundation.org/flatiron/center-for-computational-astrophysics/galaxy-formation/smaug/}{SMAUG}
collaboration}}{\footnotesize 
leading the working group for ``Resolved ISM, Star formation, and Stellar feedback'' in the international collaboration funded by the Simons Foundation entitled
``Simulating Multi-scale Astrophysics to Understand Galaxies''
%, consisting 9 PIs from 6 institutions (CCA, Princeton, Harvard, UC Berkeley, Zurich, Heidelberg) and $>$40 members. 
%The collaboration aims to build a fully predictive galaxy formation theory
%utilizing next-generation cosmological simulations 
%with physics-based subgrid models for small-scale baryonic physics.\\
%I'm leading the working group ``Resolved ISM, Star Formation, and Stellar Feedback.''
}

\elements{2017 -- 2019 }{Member, \textnormal{PICO collaboration}}{\footnotesize 
concept study for a probe mission funded by NASA entitled
``Probe of Inflation and Cosmic Origins''}

\onelineelements{2017}{Review Panelist}{NSF AAG Program}

\onelineelements{2016 -- 2017}{Organizer}{Star Formation/ISM Rendezvous Seminars at Princeton University}

\onelineelements{2012 -- }{Referee}{ApJ, ApJL, MNRAS}

%%========================================================================================
\itemtitle{Invited Reviews}

\onelineelements{2019 (planned)}{Invited Review}{\href{https://mist2019.sciencesconf.org/resource/page/id/2}{Cosmic turbulence and magnetic fields: physics of baryonic matter across time and scales}, Carg\'ese, France}

\onelineelements{2019}{Invited Review}{\href{https://www.aao.gov.au/conference/australia-eso-conference-2019}{Linking galaxies from the Epoch of initial star-formation to today}, Sydney, Australia}

\onelineelements{2016}{Invited Review}{\href{https://agenda.albanova.se/conferenceDisplay.py?confId=5696}{How Galaxies Form Stars}, Stockholm, Sweden}

\itemtitle{Invited Colloquia}

\onelineelements{2019}{Colloquium}{University of Maryland, College Park, MD}
\onelineelements{2019}{Colloquium}{Australia National University, Canberra, Austrailia}
\onelineelements{2018}{Colloquium}{Yonsei University, Seoul, Korea}
\onelineelements{2018}{Colloquium}{Korea Astronomy and Space Science Institute, Daejeon, Korea}
\onelineelements{2017}{Colloquium}{Osaka University, Osaka, Japan}
\onelineelements{2017}{Colloquium}{University of California, Santa Barbara, CA}
\onelineelements{2016}{Colloquium}{Shanghai Jiao Tong University, Shanghai, China}
\onelineelements{2016}{Colloquium}{Korea Astronomy and Space Science Institute, Daejeon, Korea}
\onelineelements{2016}{Colloquium}{Seoul National University, Seoul, Korea}
\onelineelements{2014}{Colloquium}{Korea Astronomy and Space Science Institute, Daejeon, Korea}
\onelineelements{2014}{Colloquium}{Seoul National University, Seoul, Korea}
\onelineelements{2014}{Colloquium}{Korea Institute for Advanced Study, Seoul, Korea}
\onelineelements{2011}{Colloquium}{National Institute for Mathematical Sciences, Daejeon, Korea}
\onelineelements{2011}{Colloquium}{Yonsei University, Seoul, Korea}

\itemtitle{Conference/Workshop/Seminar}

\onelineelements{2019 (planned)}{Invited Talk}{The self-organized star formation process, Orsay, France}
\onelineelements{2019}{Contributed Talk}{Feedback and its Role in Galaxy Formation, Spetses, Greece}
\onelineelements{2019}{Poster}{Linking the Milky Way and Nearby Galaxies, Helsinki, Finland}
\onelineelements{2019}{Invited Talk}{Multi-phase Gas Workshop, CCA, New York, NY}
\onelineelements{2019}{Invited Talk}{Athena++ Workshop 2019, UNLV, Las Vegas, NV}
\onelineelements{2018}{Contributed Talk}{THINKSHOP15, Potsdam, Germany}
\onelineelements{2018}{Invited Talk}{The Milky Way in the age of Gaia, Orsay, France}
\onelineelements{2018}{Invited Talk}{Kavli Summer Program in Astrophysics, CCA, New York, NY}
\onelineelements{2018}{Invited Talk}{MPPC Workshop, Princeton, NJ}
\onelineelements{2018}{Invited Talk}{CMB Foreground Workshop at CCA, New York, NY}
\onelineelements{2018}{Invited Talk}{Computational Galaxy Formation at Ringberg Castle, Germany}
\onelineelements{2017}{Invited Talk}{CMB Foreground Workshop at UCSD, San Diego, CA}
\onelineelements{2017}{Invited Talk}{The ISM beyond 3D, Orsay, France}
\onelineelements{2017}{Invited Talk}{Astrophysics Seminar, UCSB, Santa Barbara, CA}
\onelineelements{2016}{Invited Talk}{7th East-Asia Numerical Astrophysics Meeting, Beijing, China}
\onelineelements{2016}{Invited Talk}{Computational Galaxy Formation at Ringberg Castle, Germany}
\onelineelements{2015}{Contributed Talk}{Magnetic Fields in the Universe V, Carg\'ese, France}
\onelineelements{2015}{Contributed Talk}{IAU Symposium \#315, Honolulu, HI}
\onelineelements{2015}{Invited Talk}{IAS Informal Seminar, IAS, Princeton, NJ}
\onelineelements{2014}{Invited Talk}{6th East-Asia Numerical Astrophysics Meeting, Beijing, China}
\onelineelements{2014}{Invited Talk}{KITP Program -- Gravity's Loyal Opposition, Santa Barbara, CA}
\onelineelements{2013}{Invited Talk}{CITA National Fellow Meeting, Toronto, Canada}
\onelineelements{2013}{Contributed Talk}{KAS Spring Meeting, Daecheon, Korea}
\onelineelements{2012}{Invited Talk}{IAU General Assembly -- SpS12, Beijing, China}
\onelineelements{2012}{Contributed Talk}{AAS Meeting \#221, Long Beach, CA}

%%========================================================================================
\itemtitle{References}

\subtitle{Woong-Tae Kim} --
\url{wkim@astro.snu.ac.kr}, +82-2-880-6769\\
Professor, Department of Physics and Astronomy, Seoul National University 

\subtitle{Eve Ostriker} --
\url{eco@astro.princeton.edu}, +1-609-258-7240\\
Professor, Department of Astrophysical Sciences, Princeton University
%\subtitle{Greg Bryan} --
%\url{gbryan@astro.columbia.edu}, +1-212-854-6837\\
%Group Leader, Center for Computational Astrophysics, Flatiron Institute\\
%Professor, Department of Astronomy, Columbia University

\subtitle{James Stone} --
\url{jmstone@astro.princeton.edu}, +1-609-258-3815\\
Professor, School of Natural Sciences, Institute for Advanced Study

\subtitle{Snezana Stanimirovi\'c} --
\url{sstanimi@astro.wisc.edu}, +1-608-890-1458\\
Professor, Department of Astronomy, University of Wisconsin-Madison

\subtitle{Rachel Somerville} --
\url{rsomerville@flatironinstitute.org}, +1-848-445-8964\\
Group Leader, Center for Computational Astrophysics, Flatiron Institute\\
Professor, Department of Physics and Astronomy, Rutgers University

%\subtitle{David Spergel} --
%\url{dspergel@flatironinstitute.org}, +1-609-258-3589\\
%Director, Center for Computational Astrophysics, Flatiron Institute \\
%Professor, Department of Astrophysical Sciences, Princeton University

%\subtitle{Amiel Sternberg} --
%\url{amiel@astro.tau.ac.il}, 03-6407590\\
%Professor, Department of Astronomy, Tel Aviv University \\
%Senior Research Scientist, Center for Computational Astrophysics, Flatiron Institute 

Additional letters are available upon request -- 
%Woong-Tae Kim (Seoul National University, thesis advisor),
%James Stone (Princeton), 
%Snezana Stanimirovi\'c (Wisconsin), 
Greg Bryan (CCA/Columbia),
%Rachel Somerville (CCA/Rutgers), 
David Spergel (CCA/Princeton),
Shantanu Basu (Western)
Amiel Sternberg (Tel Aviv/CCA)

%
\clearpage
%%========================================================================================
\begin{center}
{\huge \bf Bibliography}\\
{\student{Name}: student primary mentored by me}\\
%\vspace{-1em}
Metrics for Refereed Publications (from \href{\adsurl}{ADS} as of \textit{2022-11-04}) \\count: 43 --- citations: 1821 --- h-index: 23
\end{center}

%\section*{Publications --- \href{\adsurl}{{\it ADS search}}}

\itemtitle{Refereed Publications}
\begin{itemize}
    \item[{43.}]\student{Kado-Fong, Erin}; \textbf{Kim, Chang-Goo}; Greene, Jenny E.; Lancaster, Lachlan, \textit{Ultra-diffuse Galaxies as Extreme Star-forming Environments. II. Star Formation and Pressure Balance in H I-rich UDGs}, \doiform{10.3847/1538-4357/ac9673}{\apj}, \textbf{939}, 101, 2022 [\href{http://adsabs.harvard.edu/abs/2022ApJ...939..101K}{2 citations}]

\item[{42.}]Kim, Jeong-Gyu; Gong, Munan; \textbf{Kim, Chang-Goo}; Ostriker, Eve C., \textit{Photochemistry and Heating/Cooling of the Multiphase Interstellar Medium with UV Radiative Transfer for Magnetohydrodynamic Simulations}, 2022, ApJS in press

\item[{41.}]Ostriker, Eve C.; \textbf{Kim, Chang-Goo}, \textit{Pressure-regulated, Feedback-modulated Star Formation in Disk Galaxies}, \doiform{10.3847/1538-4357/ac7de2}{\apj}, \textbf{936}, 137, 2022 [\href{http://adsabs.harvard.edu/abs/2022ApJ...936..137O}{8 citations}]

\item[{40.}]\student{Choi, Woorak}; \textbf{Kim, Chang-Goo}; Chung, Aeree, \textit{Ram Pressure Stripping of the Multiphase ISM: A Detailed View from TIGRESS Simulations}, \doiform{10.3847/1538-4357/ac82ba}{\apj}, \textbf{936}, 133, 2022

\item[{39.}]Motwani, Bhawna; Genel, Shy; Bryan, Greg L.; \textbf{Kim, Chang-Goo}~\textit{et al.}, \textit{First Results from SMAUG: Insights into Star Formation Conditions from Spatially Resolved ISM Properties in TNG50}, \doiform{10.3847/1538-4357/ac3d2d}{\apj}, \textbf{926}, 139, 2022 [\href{http://adsabs.harvard.edu/abs/2022ApJ...926..139M}{7 citations}]

\item[{38.}]Pingel, N. M.~\textit{et al.}~(incl. \textbf{CGK}), \textit{GASKAP-HI pilot survey science I: ASKAP zoom observations of HI emission in the Small Magellanic Cloud}, \doiform{10.1017/pasa.2021.59}{\pasa}, \textbf{39}, 2022 [\href{http://adsabs.harvard.edu/abs/2022PASA...39....5P}{3 citations}]

\item[{37.}]\student{Moon, Sanghyuk}; Kim, Woong-Tae; \textbf{Kim, Chang-Goo}; Ostriker, Eve C., \textit{Effects of Varying Mass Inflows on Star Formation in Nuclear Rings of Barred Galaxies}, \doiform{10.3847/1538-4357/ac3a7b}{\apj}, \textbf{925}, 99, 2022 [\href{http://adsabs.harvard.edu/abs/2022ApJ...925...99M}{5 citations}]

\item[{36.}]Pandya, V.~\textit{et al.}~(incl. \textbf{CGK}), \textit{Characterizing mass, momentum, energy, and metal outflow rates of multiphase galactic winds in the FIRE-2 cosmological simulations}, \doiform{10.1093/mnras/stab2714}{\mnras}, \textbf{508}, 2979, 2021 [\href{http://adsabs.harvard.edu/abs/2021MNRAS.508.2979P}{30 citations}]

\item[{35.}]\student{Lancaster, Lachlan}; Ostriker, Eve C.; Kim, Jeong-Gyu; \textbf{Kim, Chang-Goo}, \textit{Star Formation Regulation and Self-pollution by Stellar Wind Feedback}, \doiform{10.3847/2041-8213/ac3333}{\apj}, \textbf{922}, 2021 [\href{http://adsabs.harvard.edu/abs/2021ApJ...922L...3L}{10 citations}]

\item[{34.}]Clark, S. E.; \textbf{Kim, Chang-Goo}; Hill, J. Colin; Hensley, Brandon S., \textit{The Origin of Parity Violation in Polarized Dust Emission and Implications for Cosmic Birefringence}, \doiform{10.3847/1538-4357/ac0e35}{\apj}, \textbf{919}, 53, 2021 [\href{http://adsabs.harvard.edu/abs/2021ApJ...919...53C}{29 citations}]

\item[{33.}]\student{Lancaster, Lachlan}; Ostriker, Eve C.; Kim, Jeong-Gyu; \textbf{Kim, Chang-Goo}, \textit{Efficiently Cooled Stellar Wind Bubbles in Turbulent Clouds. II. Validation of Theory with Hydrodynamic Simulations}, \doiform{10.3847/1538-4357/abf8ac}{\apj}, \textbf{914}, 90, 2021 [\href{http://adsabs.harvard.edu/abs/2021ApJ...914...90L}{35 citations}]

\item[{32.}]\student{Lancaster, Lachlan}; Ostriker, Eve C.; Kim, Jeong-Gyu; \textbf{Kim, Chang-Goo}, \textit{Efficiently Cooled Stellar Wind Bubbles in Turbulent Clouds. I. Fractal Theory and Application to Star-forming Clouds}, \doiform{10.3847/1538-4357/abf8ab}{\apj}, \textbf{914}, 89, 2021 [\href{http://adsabs.harvard.edu/abs/2021ApJ...914...89L}{40 citations}]

\item[{31.}]\student{Moon, Sanghyuk}; Kim, Woong-Tae; \textbf{Kim, Chang-Goo}; Ostriker, Eve C., \textit{Star Formation in Nuclear Rings with the TIGRESS Framework}, \doiform{10.3847/1538-4357/abfa93}{\apj}, \textbf{914}, 9, 2021 [\href{http://adsabs.harvard.edu/abs/2021ApJ...914....9M}{7 citations}]

\item[{30.}]Pandya, V.~\textit{et al.}~(incl. \textbf{CGK}), \textit{First Results from SMAUG: The Need for Preventative Stellar Feedback and Improved Baryon Cycling in Semianalytic Models of Galaxy Formation}, \doiform{10.3847/1538-4357/abc3c1}{\apj}, \textbf{905}, 4, 2020 [\href{http://adsabs.harvard.edu/abs/2020ApJ...905....4P}{24 citations}]

\item[{29.}]Koo, Bon-Chul; \textbf{Kim, Chang-Goo}; Park, Sangwook; Ostriker, Eve C., \textit{Radiative Supernova Remnants and Supernova Feedback}, \doiform{10.3847/1538-4357/abc1e7}{\apj}, \textbf{905}, 35, 2020 [\href{http://adsabs.harvard.edu/abs/2020ApJ...905...35K}{10 citations}]

\item[{28.}]\textbf{Kim, Chang-Goo}; Ostriker, Eve C.; Fielding, Drummond B.; Smith, Matthew C.~\textit{et al.}, \textit{A Framework for Multiphase Galactic Wind Launching Using TIGRESS}, \doiform{10.3847/2041-8213/abc252}{\apj}, \textbf{903}, 2020 [\href{http://adsabs.harvard.edu/abs/2020ApJ...903L..34K}{16 citations}]

\item[{27.}]Gong, Munan; Ostriker, Eve C.; \textbf{Kim, Chang-Goo}; Kim, Jeong-Gyu, \textit{The Environmental Dependence of the XCO Conversion Factor}, \doiform{10.3847/1538-4357/abbdab}{\apj}, \textbf{903}, 142, 2020 [\href{http://adsabs.harvard.edu/abs/2020ApJ...903..142G}{32 citations}]

\item[{26.}]Fielding, D. B.~\textit{et al.}~(incl. \textbf{CGK}), \textit{First Results from SMAUG: Uncovering the Origin of the Multiphase Circumgalactic Medium with a Comparative Analysis of Idealized and Cosmological Simulations}, \doiform{10.3847/1538-4357/abbc6d}{\apj}, \textbf{903}, 32, 2020 [\href{http://adsabs.harvard.edu/abs/2020ApJ...903...32F}{33 citations}]

\item[{25.}]\textbf{Kim, Chang-Goo}; Ostriker, Eve C.; Somerville, Rachel S.; Bryan, Greg L.~\textit{et al.}, \textit{First Results from SMAUG: Characterization of Multiphase Galactic Outflows from a Suite of Local Star-forming Galactic Disk Simulations}, \doiform{10.3847/1538-4357/aba962}{\apj}, \textbf{900}, 61, 2020 [\href{http://adsabs.harvard.edu/abs/2020ApJ...900...61K}{52 citations}]

\item[{24.}]Seon, Kwang-il; \textbf{Kim, Chang-Goo}, \textit{Ly-alpha Radiative Transfer: Monte Carlo Simulation of the Wouthuysen-Field Effect}, \doiform{10.3847/1538-4365/aba2d6}{\apjs}, \textbf{250}, 9, 2020 [\href{http://adsabs.harvard.edu/abs/2020ApJS..250....9S}{16 citations}]

\item[{23.}]Murray, Claire E.; Peek, J. E. G.; \textbf{Kim, Chang-Goo}, \textit{Extracting the Cold Neutral Medium from H I Emission with Deep Learning: Implications for Galactic Foregrounds at High Latitude}, \doiform{10.3847/1538-4357/aba19b}{\apj}, \textbf{899}, 15, 2020 [\href{http://adsabs.harvard.edu/abs/2020ApJ...899...15M}{15 citations}]

\item[{22.}]\student{Mao, S. Alwin}; Ostriker, Eve C.; \textbf{Kim, Chang-Goo}, \textit{Cloud Properties and Correlations with Star Formation in Self-consistent Simulations of the Multiphase ISM}, \doiform{10.3847/1538-4357/ab989c}{\apj}, \textbf{898}, 52, 2020 [\href{http://adsabs.harvard.edu/abs/2020ApJ...898...52M}{15 citations}]

\item[{21.}]Kim, Woong-Tae; \textbf{Kim, Chang-Goo}; Ostriker, Eve C., \textit{Local Simulations of Spiral Galaxies with the TIGRESS Framework. I. Star Formation and Arm Spurs/Feathers}, \doiform{10.3847/1538-4357/ab9b87}{\apj}, \textbf{898}, 35, 2020 [\href{http://adsabs.harvard.edu/abs/2020ApJ...898...35K}{29 citations}]

\item[{20.}]\student{Kado-Fong, Erin}; Kim, Jeong-Gyu; Ostriker, Eve C.; \textbf{Kim, Chang-Goo}, \textit{Diffuse Ionized Gas in Simulations of Multiphase, Star-forming Galactic Disks}, \doiform{10.3847/1538-4357/ab9abd}{\apj}, \textbf{897}, 143, 2020 [\href{http://adsabs.harvard.edu/abs/2020ApJ...897..143K}{19 citations}]

\item[{19.}]\student{Vijayan, Aditi}; \textbf{Kim, Chang-Goo}; Armillotta, Lucia; Ostriker, Eve C.~\textit{et al.}, \textit{Kinematics and Dynamics of Multiphase Outflows in Simulations of the Star-forming Galactic Interstellar Medium}, \doiform{10.3847/1538-4357/ab8474}{\apj}, \textbf{894}, 12, 2020 [\href{http://adsabs.harvard.edu/abs/2020ApJ...894...12V}{20 citations}]

\item[{18.}]\student{El-Badry, Kareem}; Ostriker, Eve C.; \textbf{Kim, Chang-Goo}; Quataert, Eliot~\textit{et al.}, \textit{Evolution of supernovae-driven superbubbles with conduction and cooling}, \doiform{10.1093/mnras/stz2773}{\mnras}, \textbf{490}, 1961, 2019 [\href{http://adsabs.harvard.edu/abs/2019MNRAS.490.1961E}{40 citations}]

\item[{17.}]\textbf{Kim, Chang-Goo}; Choi, Steve K.; Flauger, Raphael, \textit{Dust Polarization Maps from TIGRESS: E/B Power Asymmetry and TE Correlation}, \doiform{10.3847/1538-4357/ab29f2}{\apj}, \textbf{880}, 106, 2019 [\href{http://adsabs.harvard.edu/abs/2019ApJ...880..106K}{28 citations}]

\item[{16.}]Murray, C. E.~\textit{et al.}~(incl. \textbf{CGK}), \textit{The 21-SPONGE H I Absorption Line Survey. I. The Temperature of Galactic H I}, \doiform{10.3847/1538-4365/aad81a}{\apjs}, \textbf{238}, 14, 2018 [\href{http://adsabs.harvard.edu/abs/2018ApJS..238...14M}{57 citations}]

\item[{15.}]Gong, Munan; Ostriker, Eve C.; \textbf{Kim, Chang-Goo}, \textit{The X CO Conversion Factor from Galactic Multiphase ISM Simulations}, \doiform{10.3847/1538-4357/aab9af}{\apj}, \textbf{858}, 16, 2018 [\href{http://adsabs.harvard.edu/abs/2018ApJ...858...16G}{43 citations}]

\item[{14.}]\textbf{Kim, Chang-Goo}; Ostriker, Eve C., \textit{Numerical Simulations of Multiphase Winds and Fountains from Star-forming Galactic Disks. I. Solar Neighborhood TIGRESS Model}, \doiform{10.3847/1538-4357/aaa5ff}{\apj}, \textbf{853}, 173, 2018 [\href{http://adsabs.harvard.edu/abs/2018ApJ...853..173K}{125 citations}]

\item[{13.}]\textbf{Kim, Chang-Goo}; Ostriker, Eve C., \textit{Three-phase Interstellar Medium in Galaxies Resolving Evolution with Star Formation and Supernova Feedback (TIGRESS): Algorithms, Fiducial Model, and Convergence}, \doiform{10.3847/1538-4357/aa8599}{\apj}, \textbf{846}, 133, 2017 [\href{http://adsabs.harvard.edu/abs/2017ApJ...846..133K}{127 citations}]

\item[{12.}]Murray, Claire E.; Stanimirovi{\'c}, Sne{\v{z}}ana; \textbf{Kim, Chang-Goo}; Ostriker, Eve C.~\textit{et al.}, \textit{Recovering Interstellar Gas Properties with Hi Spectral Lines: A Comparison between Synthetic Spectra and 21-SPONGE}, \doiform{10.3847/1538-4357/aa5d12}{\apj}, \textbf{837}, 55, 2017 [\href{http://adsabs.harvard.edu/abs/2017ApJ...837...55M}{21 citations}]

\item[{11.}]Safranek-Shrader, Chalence; Krumholz, Mark R.; \textbf{Kim, Chang-Goo}; Ostriker, Eve C.~\textit{et al.}, \textit{Chemistry and radiative shielding in star-forming galactic discs}, \doiform{10.1093/mnras/stw2647}{\mnras}, \textbf{465}, 885, 2017 [\href{http://adsabs.harvard.edu/abs/2017MNRAS.465..885S}{43 citations}]

\item[{10.}]\textbf{Kim, Chang-Goo}; Ostriker, Eve C.; Raileanu, Roberta, \textit{Superbubbles in the Multiphase ISM and the Loading of Galactic Winds}, \doiform{10.3847/1538-4357/834/1/25}{\apj}, \textbf{834}, 25, 2017 [\href{http://adsabs.harvard.edu/abs/2017ApJ...834...25K}{109 citations}]

\item[{9.}]\textbf{Kim, Chang-Goo}; Ostriker, Eve C., \textit{Vertical Equilibrium, Energetics, and Star Formation Rates in Magnetized Galactic Disks Regulated by Momentum Feedback from Supernovae}, \doiform{10.1088/0004-637X/815/1/67}{\apj}, \textbf{815}, 67, 2015 [\href{http://adsabs.harvard.edu/abs/2015ApJ...815...67K}{84 citations}]

\item[{8.}]\textbf{Kim, Chang-Goo}; Ostriker, Eve C., \textit{Momentum Injection by Supernovae in the Interstellar Medium}, \doiform{10.1088/0004-637X/802/2/99}{\apj}, \textbf{802}, 99, 2015 [\href{http://adsabs.harvard.edu/abs/2015ApJ...802...99K}{272 citations}]

\item[{7.}]\textbf{Kim, Chang-Goo}; Ostriker, Eve C.; Kim, Woong-Tae, \textit{Three-dimensional Hydrodynamic Simulations of Multiphase Galactic Disks with Star Formation Feedback. II. Synthetic H I 21 cm Line Observations}, \doiform{10.1088/0004-637X/786/1/64}{\apj}, \textbf{786}, 64, 2014 [\href{http://adsabs.harvard.edu/abs/2014ApJ...786...64K}{43 citations}]

\item[{6.}]\textbf{Kim, Chang-Goo}; Basu, Shantanu, \textit{Long-term Evolution of Decaying Magnetohydrodynamic Turbulence in the Multiphase Interstellar Medium}, \doiform{10.1088/0004-637X/778/2/88}{\apj}, \textbf{778}, 88, 2013 [\href{http://adsabs.harvard.edu/abs/2013ApJ...778...88K}{5 citations}]

\item[{5.}]\textbf{Kim, Chang-Goo}; Ostriker, Eve C.; Kim, Woong-Tae, \textit{Three-dimensional Hydrodynamic Simulations of Multiphase Galactic Disks with Star Formation Feedback. I. Regulation of Star Formation Rates}, \doiform{10.1088/0004-637X/776/1/1}{\apj}, \textbf{776}, 1, 2013 [\href{http://adsabs.harvard.edu/abs/2013ApJ...776....1K}{164 citations}]

\item[{4.}]\textbf{Kim, Chang-Goo}; Kim, Woong-Tae; Ostriker, Eve C., \textit{Regulation of Star Formation Rates in Multiphase Galactic Disks: Numerical Tests of the Thermal/Dynamical Equilibrium Model}, \doiform{10.1088/0004-637X/743/1/25}{\apj}, \textbf{743}, 25, 2011 [\href{http://adsabs.harvard.edu/abs/2011ApJ...743...25K}{125 citations}]

\item[{3.}]\textbf{Kim, Chang-Goo}; Kim, Woong-Tae; Ostriker, Eve C., \textit{Galactic Spiral Shocks with Thermal Instability in Vertically Stratified Galactic Disks}, \doiform{10.1088/0004-637X/720/2/1454}{\apj}, \textbf{720}, 1454, 2010 [\href{http://adsabs.harvard.edu/abs/2010ApJ...720.1454K}{22 citations}]

\item[{2.}]\textbf{Kim, Chang-Goo}; Kim, Woong-Tae; Ostriker, Eve C., \textit{Galactic Spiral Shocks with Thermal Instability}, \doiform{10.1086/588752}{\apj}, \textbf{681}, 1148, 2008 [\href{http://adsabs.harvard.edu/abs/2008ApJ...681.1148K}{52 citations}]

\item[{1.}]\textbf{Kim, Chang-Goo}; Kim, Woong-Tae; Ostriker, Eve C., \textit{Interstellar Turbulence Driving by Galactic Spiral Shocks}, \doiform{10.1086/508160}{\apj}, \textbf{649}, 2006 [\href{http://adsabs.harvard.edu/abs/2006ApJ...649L..13K}{44 citations}]
\end{itemize}

\itemtitle{Preprints}
\begin{itemize}
    \input{pubs_unref}
\end{itemize}

%\nocite{*}
%\bibliographystyle{plainyr-rev}
%\bibliography{mypapers}

%\vspace{10pt}

\itemtitle{Refereed Conference Proceedings}

\begin{itemize}
\item[2.] \boldname{} and E.~C. {Ostriker}, 2016, In P.~{Jablonka},
  P.~{Andr{\'e}}, and F.~{van der Tak}, editors, {\em From Interstellar Clouds
  to Star-Forming Galaxies: Universal Processes?}, volume 315 of {\em IAU
  Symposium}, pages 38--41, \emph{{Feedback Regulated Turbulence, Magnetic
  Fields, and Star Formation Rates in Galactic Disks}}.
\item[1.] \boldname{}, E.~C. {Ostriker}, and W.-T. {Kim}, March 2015,
  Highlights of Astronomy, 16:609--610, March 2015, \emph{{Numerical modeling
  of multiphase, turbulent galactic disks with star formation feedback}}.
\end{itemize}

\itemtitle{Papers in Preparation}
\begin{enumerate}
\item \boldname{}, Eve Ostriker, and the SMAUG collaboration \emph{Numerical Simulations of Multiphase Winds and Fountains from Star-Forming Galactic Disks: II.  Milky Way Analog {\it TIGRESS} Models}
%\item \boldname{} and Eve Ostriker, \emph{Numerical Simulations of Multiphase
%  Winds and Fountains from Star-Forming Galactic Disks: III.
%  Characteristics of \ion{C}{4} and \ion{O}{6} Emitting Gas in TIGRESS}
%\item \boldname{}, Yuan-Sen Ting, and Mohammad Refat, \emph{Statistics of Metallicity Fluctuations in TIGRESS}
\item \student{Woorak Choi}, \boldname{}, and Aeree Chung, \emph{Resolved Numerical Simulations of the Multiphase, Turbulence, Magnetized ISM Interacting with ICM Ram Pressure}
\item Kwang-Il Seon and \boldname{}, \emph{Lyman-alpha Radiation Transfer: I. the Wouthuysen-Field Effect}
\item Bon-Chul Koo, \boldname{}, and Sangwook Park, \emph{Radiative Supernova Remnants and Supernova Feedback}
\item \student{Aditi Vijayan}, Lucia Armillotta, \boldname{}, Eve C. Ostriker, and Miao Li, \emph{Kinematics and Dynamics of Multiphase Outflows in the solar neighborhood TIGRESS model}
\end{enumerate}
\end{document}
